\chapter{Introduction}

\section{Background} \label{sec:background}
The impressive matching of form to function in biological systems has long been admired by engineers, giving rise to the field of biomimicry, where design elements generated by the evolutionary process are employed in technological applications. Examples of biomimicry include legged locomotion in robotics that provides both efficiency and maneuverability \cite{Grimes2012THE}, nanotextures mimicking shark skin on boats that discourage barnacle growth while simultaneously decreasing water drag on the vessel \cite{Stenzel2011Drag-reducingShipping}, and tire treads inspired by the wet-adhesive properties of tree frog toe pads \cite{Persson2007WetTires}.
Soon after the advent of modern computing, researchers began experimenting with biomimicry at a higher level of abstraction. Instead of mimicking the particular phenotypic forms generated through evolution, they harnessed the evolutionary process --- repeated cycles of selection on random variation --- to generate novel solutions to a wide array of problems.
This approach has since blossomed into the field of evolutionary algorithm (EA) design \cite{Mitchell1996AnAlgorithms}.
Language used to discuss EA reflects the biological metaphor on which the algorithm is predicated.
A glossary, reviewing evolutionary concepts from both a biological and a digital perspective, is provided.
Terms that appear in the glossary are italicized upon their introduction.


Evolutionary algorithms operate on candidate solutions to a problem, which in the biological metaphor are equivalent to \glspl{individual}.
The aptitude of candidate solutions to solving a target problem is used to determine the candidate solution's \gls{fitness}, the amount of offspring it generates.
Evolutionary algorithms (EAs) traditionally begin with a \gls{population} of randomly-generated candidate solutions.
Then, through a series of successive generations, the population is regenerated through \gls{recombination} of fit candidate solutions, so \gls{selection} is performed for candidate solutions that better satisfy the target problem.
In biological evolution, a distinction is drawn between the \gls{phenotype} of an individual --- the physical characteristics which govern its interaction with the environment, its morphological, physiological, chemical, and molecular characteristics --- and the \gls{genotype} of an individual --- the heritable information that influences the phenotype displayed by the individual, i.e. the ordered sequence of base pairs in its DNA.
This distinction can become blurred in the realm of evolutionary algorithms, where the phenotypic characteristics of an individual might be directly encoded in the genotype.

The desired outcome of the evolutionary algorithm is, as generations elapse, to observe candidate solutions that provide an increasingly satisfactory solution to the target problem that was used to determine their fitness.
Once a predefined stopping criterion is met, usually after a specific number of generations or at a threshold fitness score, the evolutionary algorithm halts.
Researchers and engineers have widely demonstrated the ability of EAs to attack labor-intensive optimization problems and to discover novel solutions beyond the reach of human ingenuity \cite{Poli2008AProgramming}.
For example, evolutionary methods have been successfully applied to evolve communication antenna designs to satisfy the demanding specifications necessary for use in miniaturized spacecraft  \cite{Hornby2006AutomatedAlgorithms}.
Figure \inputandref{evolved_antenna} depicts an evolved antenna design from that project. Although its form appears alien to traditional human approaches to design, it is nonetheless effective.


\section{Defining Evolvability}

While biological phenotypic adaptation is indeed spectacular, another marvel of biology lurks just below the parade of phenotypes well-suited to their respective environments.
It is hypothesized that biological organisms exhibit adaptation to the evolutionary process itself, not just to their environment over the course of their lifespans.
That is, biological organisms are thought to possess traits that facilitate successful evolutionary search.
The term evolvability was coined to describe such traits.
A general consensus exists in the literature that evolvability stems from traits that facilitate the generation of \textit{novel} heritable phenotypic variation that is \textit{viable}.\footnote
{This statement does not suggest that mutation is nonrandom, a controversial and widely discredited theory referred to biologists as adaptive mutation.
Instead, it is predicated on the notion that the internal configuration of a biological system (i.e. the developmental process, modularity, degeneracy, etc.) constrains the outcomes of arbitrary perturbations to that system.
It is hypothesized that biological organisms possess traits that influence the distribution of phenotypic effects of random mutation.}
Evolvability can be conceived of concretely by imagining a gallery of offspring as depicted in Figure \inputandref{arabidopsis_mutants}.
Evolvability is determined by the composition of this gallery, the degree to which variation introduced by mutation is deleterious and the amount of phenotypic diversity observed among the offspring in the gallery. 
An organism among whose potential offspring exist a nontrivial number of individuals that have relatively fit phenotypic forms that exhibit significant structural diversity among themselves and in relation to their parent is highly evolvable. 

Breaking the concept down, evolvability stems from:
\begin{enumerate}
\item the amount of \textit{novel}, heritable phenotypic variation among offspring,
\item the degree to which heritable phenotypic variation among offspring is \textit{viable},\footnote
{This can be thought of in terms of the frequency at which lethal or otherwise severely harmful mutational outcomes are observed.}
\end{enumerate}
The dependence of evolution on these capacities is straightforward.
Without any heritable variation, evolution would have no raw material to select from and would stagnate.
Without any viable variation, evolution would select against all novelty and again stagnate.
Hence, systematic evolutionary change depends the production of heritable, novel phenotypic variation, some of which must not be severely deleterious.
We have established plausible traits that might facilitate evolution, but several important questions remain unanswered.
How does evolvability manifest in biological organisms (i.e. what traits of biological organisms provide explanations for the presence of viable heritable variation among offspring)?
Why does evolvability manifest (i.e. what ultimate mechanistic forces endow biological organisms with traits that promote evolvability)?
Addressing these two questions gives us a shot at tackling a third: how can evolvability be promoted in evolutionary algorithms?
We will proceed to explore these questions, but let's begin by priming our intuition for evolvability by considering an artificial selection experiment performed on \textit{Drosophila melangoster}, common fruit flies.

\section{Introductory Glimpses of Evolvability for Biologists}

Experiments performed by Tuinstra et al. (1990) and Coyne (1987) revealed that bilaterally asymmetric phenotypic traits, such as different-sized eyes, could not be induced through artificial selection.
In contrast, other artificial selection criteria, such as overall smaller eyes, yielded observable phenotypic changes over the course of a number of generations.
A cartoon summarizing these results is provided in Figure \inputandref{fly_canalization}.
The success of artificial selection for most traits on \textit{Drosophila} demonstrates the existence of a good amount of heritable phenotypic variation for those traits.
It is hypothesized that the negative result in artificial selection for bilaterally asymmetric phenotypic traits is due to a lack of bilateral symmetry-breaking information during the embryological development of \textit{Drosophila}.
In other words, the very nature of the developmental process constrains the nature of phenotypic variation that can be observed in offspring, in this case curtailing the abundance of offspring that lack bilateral symmetry. As Tuinstra et al. (1990) phrase it, ``the developmental system does not seem to allow this type of variation.''
In the life of a fly, buzzing about in search of food and sex, bilateral symmetry is usually more fit than asymmetry.
In this way, the distribution of phenotypic diversity in offspring is biased away from a particular type of deleterious variation, asymmetry.
The results from these artificial selection experiments can be cast in terms of evolvability: the distribution of phenotypic outcomes of mutation is not entirely arbitrary.
\textit{Drosophila melangoster}  more readily exhibits heritable phenotypic variation for certain traits --- overall eye size, for example --- than for other traits, such as bilateral asymmetry.

In addition to qualities that constrain against non-viable mutational outcomes, discussion of evolvability is predicated on the notion that biological organisms can possess qualities that facilitate significant heritable variation for some phenotypic trait.
The regulatory action of hormonal signals such as somatotropin exemplify such a quality. This compound, also known as growth hormone, is well known for its widespread anabolic effects on tissues throughout the body.
Mutations affecting the regulatory pathways that regulate somatotropin production and release, receptors and cell signaling components that mediate cellular response to somatotropin, and the protein itself all provide avenues for significant heritable variation in body size \cite{Devesa2016MultipleGrowth}.\footnote
{Recent research implicates somatotropin in a number of processes unrelated to its classical association with metabolism and growth. Although the phenotypic consequences of mutations affecting somatotropin pathways are not exclusively limited to body size, somatotropin response nonetheless provides an avenue for heritable phenotypic variation in that regard.}
The presence of such hormonal signaling pathways could be viewed as making a broad range of heritable phenotypic variation more readily realizable via mutation, increasing individual evolvabiltiy.
Dog breeds, which exhibit a range of body weights nearly spanning an entire order of magnitude, evidence the accessibility of heritable variation for body size in animals.
Among certain groups of dogs, much of this variation can be explained by just six genes, several of which are associated with pathways somatotropin participates in \cite{Rimbault2013DerivedBreeds.}.

\section{Introductory Glimpses of Evolvability for Computer Scientists} \label{sec:glimpses-computer-scientists}

Computer scientists who have worked on software understand that two pieces of code that meet identical specifications --- return identical output for any input given --- can differ vastly in difficulty to extend, modify, or maintain. 
Software implementation, internal structures largely invisible from the perspective of an external interface, accounts for this discrepancy.
Computer scientists use the derogatory phrase ``spaghetti code'' to describe software that is perhaps functional but implemented with such a convoluted control structure that making changes that result in a desired functional outcome becomes very difficult.
For a moment, imagine that you are a corporate executive overseeing a large collection of junior developers sourced from the local primate house to develop a word processing application.
Imagine that you prompt your army of monkeys at keyboards to begin making arbitrary changes to copies of your code base.
You might quantify the functional outcomes of these arbitrary changes.
Some changes might have no effect on the functionality of your software product; the software meets the same specifications before and after the code modification.
Some changes might fundamentally break the software product, causing it to fail to compile or crash on load.
Other changes might cause slight changes to its behavior that significantly degrade its efficacy such as fixing the text cursor at the top of the document.
Yet other changes might cause significant changes to its behavior that do not significantly degrade its efficacy such as a complete shuffling of the user menu.
It is not inconceivable that the internal configuration of your code base --- the extent to which functionality is modularized, the extent to which constants are hard coded versus declared globally, etc. --- would affect the outcomes of arbitrary changes made by your junior developers.
If the code were structured as a single source file without exception handling, arbitrary changes to the code might be expected to fundamentally break the software more frequently.
If styling information were factored out to a separate specification instead of provided individually for each element of the graphical user interface, arbitrary changes to the code might be expected more frequently to cause large non-lethal alterations to the software product by changing the styles of many aspects of the graphical user interface in one go.

The intent of this thought experiment is not to equate biological evolution and software design. 
These two processes differ fundamentally on several levels.
For example, unlike biological mutation software modifications are not performed at random.
The intention is instead to make concrete the notion that internal system configurations fundamentally constrain the outcomes of perturbation of the system, be it through source code changes or mutation.
Computer scientists encountering difficulty concretely envisioning how evolvability might manifest in biological systems --- or skeptical of the ability of internal system configuration to influence the outcomes of mutation --- might take a few moments to recall their own experiences with ``spaghetti code.''

\section{Synopsis}

Before embarking on our exciting expedition through evolvability theory as it manifests in biological and computational domains, it would be wise to spend a moment consulting the travel agency's promotional brochure to preview the trip's itinerary and anticipate the souvenirs we might hope to take away.
While the concept of evolvability, promotion of viable heritable variation, is straightforward, understanding how and why evolvability arises in biological organisms is much more complicated.
As will be evident, evolvability can not be traced through a set of direct relationships back to a single point of origin; it appears instead to be an artifact of multiple causality.
The causal roots of evolvability branch out in a dizzying multitude of directions.

The primary aim of this work is to survey, organize, and analyze factors that contribute to evolvability using evidence and theory that cuts across the fields of evolutionary computing and evolutionary biology.
A novel conceptual framework for the causal factors related to evolvability,
which are grouped into proximate, intermediate, and ultimate categories, is presented.
This framework is inspired by the distinction between proximate causality, the immediate physical explanation for a phenomenon, and ultimate causality, the less immediate evolutionary explanation for a phenomenon, fundamental to traditional evolutionary thinking.
Returning to the conceptual framework presented for evolvability, the label proximal is applied to causal factors of evolvability related to how the physical forms and processes of biological organisms dictate the types of phenotypic variation that can be realized.
Discussion of intermediate causal factors related to evolvability steps back to a slightly greater level of abstraction.
Intermediate causal factors might be succinctly described as design principles.
More concretely, they are overarching patterns in the internal structure and processes of biological systems, manifesting in diverse contexts across and within different levels of biological organization.
Observations of these overarching patterns are typically not limited to just their absence or presence, but often can describe the degree to which they are observed.
Proximal and intermediate factors can be remarked upon by observing the properties of an individual in isolation from the larger evolutionary process.
Ultimate causality focuses on the fundamental causal factors related to evolvability that give rise to intermediate and proximate causality.
From an EA perspective, ultimate causal factors would reflect fundamental assumptions explicitly built into a model of biological evolution while intermediate and proximate causal factors would not be explicitly accounted for in the model but would instead be expected to emerge in the model under appropriate conditions.
A review of causal factors related to evolvability, organized by the framework categories of proximate, intermediate, and ultimate, is laid out in Chapters \ref{sec:proximate}, \ref{sec:intermediate}, and \ref{sec:ultimate}.

In Chapter \ref{sec:discussion}, the proximate-intermediate-ultimate framework is leveraged to analyze possible strategies to promote evolvability.
This analysis demonstrates how modifications to both major components of the evolutionary algorithm, the genotype-phenotype mapping and the phenotype-fitness mapping, can be achieved in a manner consistent or inconsistent with the biological metaphor.
It is suggested that biologically implausible efforts to promote evolvability at the proximate and intermediate causal levels provide an avenue to achieve evolvability at a feasible computational cost.
These strategies to promote evolvability in evolutionary computing discussed in this section are well trod, and lucratively so.
This analysis aims not to unearth of novel strategies to promote evolvability in EA, but instead to provide an overarching theoretical explanation and contextualization of these strategies.

Discussion of biologically plausible versus biologically implausible intervention strategies nicely underlines the crux of evolutionary computing: what level of biological realism is necessary to realize the salient features of evolution?
This train of thought arrives near the conclusion of our journey through evolvability.
Chapter \ref{sec:conclusion} reflects on the issue of evolvability to more broadly situate the fields of evolutionary computing and evolutionary biology in relation to one another.
That crux of evolutionary computing can be recast as the question of what components of an evolutionary model are necessary in order to observe innovation on par biological evolution.
It is argued that building and evaluating models shifts the task of evaluating the completeness of any particular confection of evolutionary theory from a qualitative endeavor towards a more quantitative endeavor.
Evolvability itself crystallizes this point exactly.
Without rigorous modeling, the imperative to appeal to evolvability for explanatory purposes is nebulous (although nonetheless recognized by some elements of the evolutionary biology community).
Presenting evolution as a narrative tracing the interactions of mechanistic players, the structure of phenotypic variation can be plausibly dismissed as merely random or otherwise left unconsidered.
In contrast, lackluster evolvability exhibited by naive models of evolution has loomed as a very concrete roadblock to evolutionary computing researchers; attempting to model evolution makes stark the problem of evolvability.

To summarize the road map compactly, Chapter \ref{sec:definition} will equip us with vocabulary and metrics necessary to precisely quantify and characterize evolvability.
Chapters \ref{sec:proximate}, \ref{sec:intermediate}, and \ref{sec:ultimate} will review putative factors contributing to evolvability, illustrated with biologically-motivated examples and organized into a hierarchy from proximal to intermediate to ultimate.
Chapter \ref{sec:discussion} will return focus to evolutionary computing, analyzing the factors related to evolvability to examine two overarching intervention strategies to promote evolvability in digital models of evolution.
A handful of illustrative examples of these two strategies in action drawn from the evolutionary computing literature follows this exposition.
Chapter \ref{sec:conclusion} concludes the paper by reflecting on the relation between evolutionary biology and evolutionary computing.
Evolvability specially exemplifies the necessity of modeling to rigorously evaluate the explanatory power of any synthesis of evolutionary theory.
The overall goal of this work is to draw a fresh, rigorous connection between theory and major trends, and promising future directions, in evolutionary computing.

With the promotional pamphlet exhausted and the itinerary set, we are now well prepared to embark.
