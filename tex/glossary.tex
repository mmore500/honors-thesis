\makeglossaries
 
\newglossaryentry{individual}
{
    name=individual,
    description={Individuals are the object upon which evolution operates; evolution evaluates and selects on individuals and recombines individuals to form new individuals. In biology, and individual is an individual organism such as a single tree or a single bird. In evolutionary algorithms, an individual is abstracted as a candidate solution to a problem.}
}
 
\newglossaryentry{population}
{
    name=population,
    description={A population is a collection of individuals that compete to transmit their genetic information to the next generation. These individuals are typically highly similar and, in many cases in both biology and evolutionary algorithms, recombine their genetic information to produce offspring.}
}

\newglossaryentry{phenotype}
{
    name=phenotype,
    description={Phenotype refers to the characteristics of an individual that interact with its environment to determine its fitness. In biology, the physical form of an organism (i.e. its body) is the phenotype. In evolutionary algorithms, the phenotype refers to the characteristics of an individual that are evaluated during selection.}
}

\newglossaryentry{genotype}
{
    name=genotype,
    description={Genotype refers to information that is used to determine the phenotype that is passed from generation to generation. In biology, a DNA sequence serves as the genotype. Although many different genotypic encodings are employed in evolutionary algorithms, in evolutionary algorithms the genotype ultimately boils down to a collection of digital information.}
}

\newglossaryentry{recombination}
{
    name=recombination,
    description={Recombination refers to the generation of new genetic material from existing genetic material. This can involve combinations of two or more sets of genetic material, as in sexual reproduction, and/or random perturbation of genetic information (i.e. mutations).}
}

\newglossaryentry{selection}
{
    name=selection,
    description={Selection refers to the determination of which individuals will pass genetic material on to the next generation by creating offspring (and how many offspring they will be able to generate) and which will not.}
}

\newglossaryentry{fitness}
{
	name=fitness,
    description={Fitness refers to the success of an individual at passing its genetic information to the next generation. An individual with high fitness creates many offspring while an individual with low fitness does not. Success at surviving challenges posed by the environment is an important factor in determining fitness. In evolutionary algorithms, the concept of fitness is abstracted to the fitness function where an individual is scored based on its aptitude at performing a certain task.}
}