\chapter{Making Evolvability Concrete} \label{sec:definition}

\section{Characterizing Evolvability} \label{sec:characterizing}

In relation to both aspects of evolvability, generation of heritable variation and bias towards heritable variation, researchers have developed further theoretical distinctions that allow evolvability to be discussed in a more nuanced and concrete manner. Focusing on evolvability as the generation of heritable variation aspect, discounting bias towards viable variation, Wilder et al. describe two measures of evolvability: individual evolvability and population evolvability \cite{Wilder2015ReconcilingEvolvability}. Individual evolvability refers to the potential of an individual to yield a set of offspring that exhibit phenotypic diversity. Figure \inputandref{high_vs_low_individual_evolvability} contrasts high and low individual evolvability. In contrast, population evolvability refers to total amount of phenotypic diversity among potential offspring of a population as a whole \cite{Wilder2015ReconcilingEvolvability}. These two measures of evolvability are contrasted in Figure \inputandref{individual_vs_population_evolvability}. Although individual and population evolvability might be correlated to some extent, there is not a direct relationship between the two. As Wilder et. al admonish, ``population-level evolvability is not equal to the sum over individual evolvability because the novel phenotypes contributed by different individuals may be redundant.'' \cite{Wilder2015ReconcilingEvolvability} The difference between these two types of evolvability is more than semantic; it is thought that population-level evolvability is a much stronger indication of the ability of an evolutionary process to widely explore its search space, discover adaptive variability, and, ultimately, to generate highly-adapted individuals. Wilder et. al argue this point convincingly,
\begin{displayquote}
``On the one hand, evolvable individuals are more likely than others to introduce phenotypic variation in their offspring. On the other hand, in evolvable populations a greater amount of phenotypic variation is accessible to the population as a whole, regardless of how evolvable any individual may be in isolation'' \cite{Wilder2015ReconcilingEvolvability}
\end{displayquote}
Population evolvability and individual evolvability stem from a different set of proximal causes. An individual with high individual evolvability occupies a region of the genotypic space that maps to a highly variable set of phenotypes; thus, a highly diverse set of phenotypes may be easily reached via small changes in the genotype space. In contrast, high population evolvability is likely due not just to the positioning of individuals in the genotype space, but also to the spreading of individuals from one another throughout the genotype space. In other words, a diverse set of parents will generate a diverse set of offspring.


Considering the other aspect of evolvability, focusing exclusively on bias towards viable variation, another theoretical distinction can be made between innate evolvability, latent evolvability and acquired evolvability. The terms latent evolvability and acquired evolvability were introduced by Reisinger et al. in \cite{Reisinger2005TowardsEvolvability} to discuss canalization, the ability of a population to control the variability generated among its offspring to in order to exploit fitness biases specific to its environment. Recalling the set of experiments from \cite{Tuinstra1990LackDevelopment,Coyne1987LackMelanogaster} reported in Section \ref{sec:background} and summarized in Figure \ref{fig:fly_canalization}, the observed bias that maintains bilateral symmetry among \textit{Drosophilia melangoster} is a form of canalization. Distinguishing between innate, latent, and acquired evolvability, it is key to observe that canalization is a ``learned'' bias, developed over the course of evolution in response. In the case of \textit{Drosophilia melangoster}, the canalization is due to the lack of symmetry breaking information in the developmental process, which itself is defined by the genome of \textit{Drosophilia}. Thus, the information enabling canalization is stored in the genome. As Reisinger et. al put it, ``evolvability emerges over the course of evolution with a specific fitness function, and is defined within the terms of that function'' \cite{Reisinger2005TowardsEvolvability}. To better describe the learned nature of canalization, Reisinger et al. introduce the differentiation between latent evolvability and acquired evolvability. According to Reisinger et al., latent evolvability describes ``the representation’s underlying capacity for becoming evolvable'' while acquired evolvability describes ``evolvability learned in response to a particular fitness function'' \cite{Reisinger2005TowardsEvolvability}. In their experiments, acquired evolvability, which can be observed and quantified, is used as a proxy for latent evolvability. I introduce the term innate evolvability to describe bias towards viable variation that is inherent to a representational scheme. For example, Clune et al. identify bias towards phenotypic regularity, which in certain environments tends to be a useful trait, as an inherent trait of indirect genetic encoding \cite{Clune2008HowDecreases}. (The relationship between indirect encoding and phenotypic regularity is discussed in detail in Section \ref{sec:regularity}). To summarize, latent evolvability describes a representational scheme's potential to support canalization. Acquired evolvability describes actual canalization exhibited by an evolving population in response to a particular fitness environment. Innate evolvability refers to nonlearned bias towards viable variation. These distinctions emphasize the fact that bias towards viable variation can arise through canalization, which is a learned trait where learning is enabled by the representational scheme that relates genotype and phenotype, or can result from qualities innate to a representational scheme, such as a bias towards phenotypic regularity.

\section{Quantifying Evolvability} \label{sec:quantifying}

Techniques to quantify evolvability are necessary to study it empirically. Reflecting the plurality of definitions for the term in the literature, several of measures of evolvability have been designed. These metrics are in no way mutually exclusive. In fact, they might be viewed as complementary as several measure distinct aspects of evolvability. Techniques to quantify evolvability relevant to evolutionary computing only, not biology, will be reviewed here.

The maximum fitness value achieved by an evolving system provides a convenient and straightforward, but blunt, assessment of evolvability. While evolvability is an important factor in successful evolutionary search, such fitness-based measures do not directly assess the the potential of an evolutionary system to generate viable variation. Thus, such a measure is not directly predictive of properties associated with evolutionary systems with high evolvability, such as the ability to adapt to new, untested environments \cite{Tarapore2015EvolvabilityBenchmarks}.

Reisinger et al. employ an evolvability measure that specifically targets acquired evolvability, bias towards viable variation learned in response to the fitness structure of a particular environment\cite{Reisinger2005TowardsEvolvability}. In their words, their measure targets ``the representation’s ability to retain and generalize information learned about a changing domain.'' Essentially, Reisinger et al. assesses evolvability as the correlation between the bias towards viable variation exhibited by an evolving system and the amount of information on the fitness structure of an environment the representation receives. Reisinger et al. work with temporally varying fitness functions, where fitness criteria are adjusted from generation to generation. Importantly, these adjustments are made to preserve invariant properties across all fitness criteria. In the bit-string domain employed by Reisinger et al., for example, all fitness criteria manifest bilateral symmetry. The amount of information broadcast to the evolving system is controlled by varying the target drift rate, the rate at which changes to the fitness function are made. Maximal information on the invariant properties of a fitness environment is broadcast at a middling rate of fitness criteria adjustment; adjusting fitness criteria too quickly amounts to training the evolving system on ``random noise'' and adjusting them too slowly amounts to training the evolving system under a nearly static evaluation scheme where insufficient evolutionary pressure to learn the invariant properties of a fitness environment is exerted. Reisinger et al. measure the assess the amount of bias towards viable variation learned by an evolving system by measuring the efficiency with which an evolving system trained in a fitness environment with invariant properties (i.e. bilateral symmetry) adapts to entirely novel fitness criteria that maintain those invariant properties. It is contended that in a highly evolvable system, maximum bias towards viable variation should be observed at a middling rate of fitness function change. A numerical measure of evolvability is thus calculated as the variance of adaptation over a range of target drift rates encountered during training.

Mengistu et al. develop a measure for individual evolvability, relying upon the characterization of evolvability as heritable variation, to power their evolvability search method \cite{Mengistu2016EvolvabilityIt}. To assess individual evolvability, a sample of an individual's offspring are generated. A Euclidean distance measure in the phenotype space is used to bin the generated offspring. Individuals exhibiting phenotypic distance from the representatives of each pre-existing bin exceeding a pre-defined, domain-specific threshold are used to found a new bin. The number of bins generated by this process, essentially a count of distinct behaviors present among a sample of the individual's offspring, serves as the metric of evolvability. Wilder et al. take a parallel approach to Mengistu et al., but sample offspring from an entire population instead of an individual in order to measure population evolvability instead of individual evolvability \cite{Wilder2015ReconcilingEvolvability}.

Clune et al. assess evolvability by examining the relationship between parental fitness and offspring fitness \cite{Clune2011OnRegularity}. Their approach evolvability provides a window into the amounts of both viable variation and (to a lesser degree) overall heritable phenotypic variation present among offspring. The fitness impacts of variation, how much is how helpful or harmful, are assessed by examining the proportion of offspring with fitness greater than (or less than) their parents and, using reasonable threshold values, the proportion of offspring with fitness much greater than (or much less than) their parents. The spread of differences between parental and offspring fitness is taken as a proxy for the diversity of phenotypic form among offspring. In this scheme, a wide spread of fitness changes would indicate high individual evolvability. While phenotypic variability and fitness variability are certainly correlated to some extent, the distinction between the two should be noted; this measure provides at best an indirect measure of phenotypic variability among offspring.

Tarapore et al. recently introduced an evolvability measure that attempts to take a more clear-eyed view of both of the primary aspects of evolvability: the amount of heritable variation among offspring and the fitness effects of that variation \cite{Tarapore2015EvolvabilityBenchmarks}. They forgo use of a scalar metric to describe evolvability, instead reporting evolvability using what they term a ``signature.'' Essentially, the signature is a two-dimensional heatmap presenting the changes in phenotypic form and fitness observed in individual offspring from a single parent. Normalized mutual information between the phenotypic states of parent and offspring is used to quantify the amount of change in phenotypic form observed in an offspring. Proportion decrease in fitness is used to quantify the fitness difference between parent and offspring. For a highly evolvable individual, we would expect to see offspring occurring with significant frequency in the corner of the heatmap indicating significant change in phenotypic form with slight or no loss of fitness. The evolvability signature provides a nuanced snapshot of evolvability, allowing for interaction between the two primary components of evolvability to be visualized. Such information can be highly diagnostic, for example alerting researchers to phenomena that might appear falsely promising using other metrics, such as genetic changes that alter phenotypic form significantly but at great cost to fitness or genetic changes that are beneficial to fitness but fail to uncover novel phenotypic form.

\section{Explaining Evolvability} \label{sec:explaining}

Considering the evolutionary process through the lens of evolvability, one realizes that the value of a solution in terms of the ultimate outcome of evolutionary search isn't completely described by its fitness scores. Although high-scoring individuals are the ultimate goal of the EA process, during the EA process individuals that have the potential to lead to high fitness offspring being discovered by the evolutionary process are the most desirable (although they themselves might not have excellent fitness scores). The fitness score is simply a rough proxy for the quality of future offspring. It does not explicitly take into account an individual's potential (or lack of it) for innovation and adaption in evolutionary time. This observation raises a quandary: how can natural selection ``favor properties that may prove useful to a given lineage in the future, but have no present adaptive function'' \cite{Pigliucci2008IsEvolvable}? Researchers are searching out explanations in two main areas: 
\begin{itemize}
\item evolutionary selection mechanisms, positing that perhaps forces beyond traditional static selection such as divergent selection or a fluctuating fitness function might encourage evolvability, and
\item developmental mechanisms, positing that indirect encoding of the phenotype adds inherent bias towards regular, modular phenotypes or allows for the introduction of learned biases that canalize mutational effects towards selectively-advantageous ends.
\end{itemize}
In biological parlance, these hypotheses fall under the umbrella of ultimate causality, factors that are considered root causes of a phenomenon. Biologists also pursue proximal explanations, which appeal to immediate physical properties and processes to account for observations, to understand phenomena they encounter. In investigating why an flower is purple, for example, a biologist may discuss refractive properties of biochemical compounds present in the flower's skin, proximate causation, as well as the fact that the flower's color scheme has been evolutionarily selected for to attract pollinators, ultimate causation. 
Our discussion of factors that promote evolvability will trace a similar trajectory. Chapter \ref{sec:proximate} will discuss proximate factors that promote evolvability in biological systems and Chapter \ref{sec:ultimate} will discuss ultimate explanations for biological evolvability. A third section, Chapter \ref{sec:intermediate}, presents a collection of properties commonly associated with biological systems thought to be related to evolvability, which occupy an intermediate ground between ultimate and intermediate. Recalling the superficial analogy drawn between software and biology in Section \ref{sec:glimpses-computer-scientists}, one might liken proximate causal factors to specific design patterns in software that promote maintainability such as the exclusive use of return statements at the end of a function, adherence to don't-repeat-yourself, or avoidance of hard-coded values. Intermediate factors could be likened to high-level attributes of software such as modularity and robustness. Finally, ultimate factors could be related to mechanisms or structures external to a software package that influence its structure such as SCRUM or Agile processes, firm culture, or the operating system it is designed to run on. In Chapter \ref{sec:discussion}, we will conclude with a graphical summary and discussion of how high-level factors that promote evolvability (i.e. intermediate and ultimate causes) interact.
