\chapter{Intermediate Causes of Evolvability} \label{sec:intermediate}

Discussion of intermediate causal factors related to evolvability steps back to a slightly greater level of abstraction than proximal causal factors.
These causal factors describe design motifs common to biological systems rather than specific processes or structures as was the case in discussion of proximal causal factors.
The overarching patterns in the internal structure and processes of biological systems classified as intermediate causal factors manifest in diverse contexts across and within different levels of biological organization.
Observations of intermediate causal factors related to evolvability are typically not limited to just their absence or presence, but often can describe the degree to which they are observed.
That is, these intermediate causal factors related are often quantifiable, especially in a digital context.

These causal factors are termed intermediate because meaningful levels of causality exist above them.
These higher level factors are termed ultimate causal factors related to evolvability and are presented in the following chapter.
For the most part, intermediate causal factors describe qualities of a biological system as a whole instead of a describing a specific subcomponent or subsystem of the system as tended to be the case for proximate factors related to evolvability.
By providing convenient language to describe causal factors related to evolvability at the level of entire organisms rather than their constituent subsystems and internal processes, intermediate causal factors yield a useful conceptual interface between proximate and ultimate causality.

\section{Modularity} \label{sec:modularity}
\subsection{Definition}
Modularity refers to the organization of a system into distinct subsystems where interaction between components of the subsystem outstrips interaction between components inside of a subsystem and components outside of it \cite[p 207]{Downing2015IntelligenceSystems}.
By distilling the system into compact units of independent functionality, modularity allows reuse of that functionality in different parts of the system through duplication \cite{Reisinger2005TowardsEvolvability}.

\subsection{Relation to Evolvability}
Downing argues convincingly for the utility of modularity,
\begin{displayquote}
Once structures or mechanisms are consolidated and isolated (to some degree) from external influences, their probability of disruption declines and their potential for self-modification without global repercussions increases: they enhance both robustness and adaptability \cite[p 208]{Downing2015IntelligenceSystems}
\end{displayquote}
The compartmentalization of functionality makes independent changes to one aspect of functionality possible while other aspects of functionality are preserved.
By increasing the robustness of other aspects of functionality in an organism to changes in one aspect of functionality, modularity ultimately biases evolutionary search space towards viable offspring.
Further, compartmentalization of functionality allows for re-use of that functionality in another context in the system.
Because complete units of phenotypic functionality are more likely to be useful in new contexts than randomly generated structures, modularity allows for the generation of divergent phenotypes (i.e. diverse offspring) that are likely to be viable. In this way, modularity promotes complexification.

\subsection{Example}
The segregated cell roles in multicellular organisms provides an excellent example of modularity.
Consider, in particular, simple squamous epithelial tissue.
This tissue cells are typically constitutes regions of membrane requiring rapid diffusion or filtration.
The tissue's self-contained structure facilitates the reuse of its functionality across many different parts of the body.
Simple squamous epithelial tissue appears in a diverse set of anatomical contexts including the air sacs of lungs, the walls of capillaries, and the kidneys \cite{Owens2003TheKeratins}.

\section{Robustness} \label{sec:robustness}
\subsection{Definition}
Robustness generally refers to the ability of a system's function to persist under perturbation. Biological systems exhibit robustness on three levels, as identified in \cite{Richter2015EvolvabilitySurvey}:
\begin{enumerate}
    \item system functionality is not degraded by stochastic fluctuations in the system, \label{item:robustness_stochastic}
  	\item phenotypic traits are generally not degraded by genetic variation, and \label{item:robustness_genetic}
    \item the phenotype is able to function if environment changes. \label{item:robustness_environment}
\end{enumerate}

\subsection{Relation to Evolvability}
By reducing outright mortality and other phenotypic variations with catastrophic fitness implications, robustness biases evolutionary search towards viable variation.

\subsection{Example}
The heart provides a prime example of level \ref{item:robustness_stochastic} robustness. The waves of electric potential that govern the function of the heart are generally robust to perturbation; the heart can usually recover from momentary disruptions to these waves and resume normal function after a short time. However, the heart is not a totally robust organ: it can be susceptible to ventricular fibrillation, where normal heart function is essentially halted by the emergence of a spiral wave pattern of electrical activation [cite]. In many cases, the consequences of the heart not being able to recover from the perturbation that induced the spiral wave pattern of activation are fatal.

Duplicate genes exemplify robustness at level \ref{item:robustness_genetic} \cite{Gu2003EvolutionMutations}. It has been shown that deletion duplicate genes have a significantly lower proportion of fatal outcomes and a significantly higher proportion of weak or no effect outcomes in \textit{Saccharomyces cerevisiae} (yeast) \cite{Gu2003EvolutionMutations}. Robustness is also provided by alternate metabolic pathways and regulatory networks that can compensate for the absence of a gene \cite{Gu2003EvolutionMutations}.

Finally, robustness at level \ref{item:robustness_environment} is exhibited by the brown rat (\textit{Rattus norvegicus}). This creature has been wildly successful in a wide range of environments --- today, its range spans nearly the entire globe --- that differ vastly from the environment where it originated \cite{Wikipediacontributors2016BrownRat}. In fact, \textit{Rattus norvegicus} is particularly notorious for thriving in urban areas, a completely novel environment that only came into existence in their current form over the last millennium or so. The brown rat displays impressive robustness to environmental variation.

\section{Canalization} \label{sec:canalization}
\subsection{Definition}
Canalization refers to ``organizing the effects of mutations such that some features become more resistant to change'' \cite{Reisinger2005TowardsEvolvability}.
Canalization can stem from many aspects of phenotypic form and the phenotype-genotype mapping.
For example, developmental constraint is associated with canalization.
Importantly, canalization can be selective --- some phenotypic features become more resistant to change than others.

\subsection{Relation to Evolvability}

Canalization can provide selective bias against inviable phenotypic outcomes while still allowing for viable phenotypic variability.
Thus, canalization contributes directly to one of the major components of evolvability: bias towards viable variation.
Through this bias towards viable variation, canalization contributes to an increase in the overall quality of offspring to be selected from \cite[p 40]{Downing2015IntelligenceSystems}.

\subsection{Example}
Developmental constraint is a form of canalization.
The developmental constraint revealed by artificial selection experiments performed on \textit{Drosophila melanogaster} illustrate canalization well.
These experiments, presented in Section \ref{sec:fly_symmetry}, revealed heritable bilaterally symmetric phenotypic variation is more readily accessible than bilaterally asymmetric variation.
Hence, \textit{Drosophila} exhibits canalization against systematic asymmetry.

\section{Direct Plasticity} \label{sec:direct_plasticity}
\subsection{Definition}
Plasticity, in general, refers to environmental influence on the phenotype. In biology, environmental and genetic influences, together, shape the phenotype. Environmental influences may alter the trajectory of the developmental process or may otherwise induce phenotype changes in response to environmental stimulus \cite{Fusco2010PhenotypicConcepts}.

Direct plasticity plasticity is specifically related to environmental influence that is exerted directly and coercively on developmental or physiological processes. An organism that exhibits direct plasticity is robust to these environmental perturbations. At a fundamental level, successful direct plasticity entails \textit{resistance} to environmental influence on the phenotype.


\subsection{Relation to Evolvability}

It is thought that the homeostatic mechanisms that mediate an organism's interactions with its environment (i.e. the mechanisms that support to direct plasticity) might also promote robustness to mutation \cite{Moczek2011TheInnovation}. That is, the same traits that help an individual maintain functionality under environmental perturbation may also help protect the individual against catastrophically deleterious mutational outcomes.

\subsection{Example}
Osmotic pressure, generated via the diffusion of water into a cell due to a disparate solute concentrations inside and outside a cell, constitutes a grave threat to biological cells.
Unchecked, this coercive physical influence may indeed burst a cell, such as an animal erythrocyte \cite{Lodish2000OsmosisVolume}.
Such cells, unable to withstand disruptive environmental physical influences, would be said to exhibit poor direct plasticity in that regard.
Many biological organisms do, however, exhibit strong direct plasticity with respect to osmotic pressure.
Plant, algal, fungal, and bacterial cells use a rigid cell wall to withstand osmotic pressures.
Many protozoa cope with osmotic influx of water by capturing excess fluid in the cytosol in a contractile vacuole  that is occasionally discharged into the extracellular environment \cite{Lodish2000OsmosisVolume}.

\section{Indirect Plasticity} \label{sec:indirect_plasticity}

\subsection{Definition}

Indirect plasticity is specifically related to environmental influences more akin to informational signals than coercive physical influences. An organism in which these environmental signals prompt responses that are mediated by physiological or developmental systems exhibit indirect plasticity. This term describes the ability of an organism to receive, process, and respond to environmental information through adjustments to its own phenotype \cite{Fusco2010PhenotypicConcepts}. At a fundamental level, successful indirect plasticity entails strategic \textit{amplification} of certain aspects of environmental influence on the phenotype.

\subsection{Relation to Evolvability}
Researchers have suggested that phenotypic modularity might be a key contributor to indirect plasticity, especially in plants \cite{Schlichting1986ThePlants, DeKroon2005APlants}. Thus, selection for indirect plasticity, which is known to play an important role in adaptation to an unpredictable or variable environment \cite{Fusco2010PhenotypicConcepts}, might promote modularity.

Conditional expression of phenotypic traits through plasticity allows for relaxed selection on the genotypic locus determining those traits. Thus, significant genetic variation can accumulate at that locus in a population. In a process known as genetic accommodation, the environmental influence on when rarely-expressed phenotypic traits are expressed can be diminished or erased through sensitizing mutation; what once was induced via environmental signal can become constitutive. Such processes have been observed experimentally via artificial selection \cite{Moczek2011TheInnovation}.

Finally, the role plastic processes such as learning might play in concert with phenotypic regularity is an open question. It is possible that phenotypic remodeling induced via environmental cues may provide a mechanism of irregular refinement of highly regular phenotypic structures generated via development. That is, interaction of an individual with its environment may induce irregular phenotypic adjustments which increase fitness \cite{Clune2011OnRegularity}.

\subsection{Example}

Tadpoles of \textit{Spea multiplicata}, which is also known as the Mexican spadefoot toad, develop along two alternate trajectories in response to environmental signals. The two tadpole morphs of \textit{Spea multiplicata}, depicted in Figure \inputandref{spea_multiplicata} exhibit different jaw and digestive tract configurations and prefer different diets \cite{Fusco2010PhenotypicConcepts}. One morph is specially suited to an carnivorous diet while the other is suited to an omnivorous one\cite{Pfennig1992PolyphenismStrategy}.

While \textit{Spea multiplicata} illustrate phenotypic plasticity in development, phenotypic plasticity --- although often more subtle --- also manifests on an ongoing basis throughout the lifespan of most creatures. \textit{E. coli} provide a textbook example of phenotypic alteration in response to environmental stimulus. These bacteria selectively express an enzyme used in the digestion of lactose, $\beta$-galactosidase in the presence of lactose when alternate food sources, such as glucose, are unavailable. This phenotypic change is mutable; in the absence of lactose or the presence of glucose production of $\beta$-galactosidase halts \cite{Griffiths2015IntroductionAnalysis}.


\section{Intraindividual Degeneracy} \label{sec:intraindividual_degeneracy}

\subsection{Definition}

Degeneracy refers to the exhibition of similar functionality by structurally different systems \cite{Edelman2001DegeneracySystems}. For example, one might refer to push doors, rotating doors, and sliding doors as degenerate; although they all accomplish similar tasks, they are structurally dissimilar.

In an organism exhibiting intraindividual degeneracy, distinct components with different structures perform the same function. Degenerate components are functionally, but not structurally, redundant.

\subsection{Relation to Evolvability}
Intraindividual degeneracy is thought to be key to the evolvability observed in biology. Intraindividual degeneracy provides a solution to the seemingly paradoxical relationship between robustness (Section \ref{sec:robustness}) and evolutionary innovation, the ability to generate heritable novel variation \cite{Whitacre2010Degeneracy:Evolvability}. How can a system have both robustness, which constrains change to the system, and innovation? Intraindividual degeneracy seems to be the answer. By having multiple independent components that perform the same task, the system is secured against the failure of one of the components and, therefore, more likely to persist under extreme environmental conditions due to the diverse response of the degenerate components to those conditions \cite{Whitacre2010Degeneracy:Evolvability}. However, intraindividual degeneracy also provides a diverse set of phenotypic structures that can be repurposed or elaborated upon. Essentially, intraindividual degeneracy protects against loss of function mutations, which are often deleterious, while providing ample jumping off points to generate novel phenotypic function. In this way, intraindividual degeneracy promotes individual evolvability.

Intraindividual degeneracy is also thought to be related to direct plasticity. Under different circumstances --- such as unusual environmental stress --- the functionality of degenerate components might diverge due to their structural differences  \cite{Richter2015EvolvabilitySurvey}. The presence of degenerate components, each of which might be able to maintain functionality under environmental stresses that would neutralize the other components, increases the range of environmental perturbations that a organism might be able to withstand.
 
\subsection{Example}
Mammalian deoxyribonucleoside kinases serve as a prime example of intraindividual degeneracy.
Mammalian deoxyribonucleoside kinases are enzymes that process the precursors to DNA.
As shown in Figure \inputandref{intraindividual_degeneracy}, these enzymes act on four substrates: dT, dC, dA, and DG.
These kinases are said exemplify degeneracy because each unit of functionality, the chemical processing of each of the four precursors, is accomplished by two structurally unique components. Processing of the compound dT, in particular, is accomplished via both TK1 and TK2, enzymes from different protein families that fundamentally differ in structural composition. The ability of an organism to metabolize the dT DNA precursor is thus more robust to perturbations that disable one of Tk1 or Tk2. Additionally, Tk1 and Tk2 both serve as jumping-off points for evolutionary innovation via the repurposing of either enzyme.

\section{Interindividual Degeneracy}

\subsection{Definition}

I define interindividual degeneracy as the presence of phenotypically dissimilar individuals in a population that, despite those dissimilarities, interact with their environment in identical or near-identical ways (i.e. are functionally equivalent).
Interindividual degeneracy is related to the idea of diversity but specifically refers to phenotypic diversity that is functionally interchangeable.
Although structurally divergent, individuals in a population that exhibits interindividual degeneracy are functionally similar.

Interindividual degeneracy can be seen as stemming from tolerance of the phenotype-fitness mapping for certain types of phenotypic variation.
That is, certain aspects of phenotypic form are inconsequential to fitness.

Interindividual degeneracy can also be seen as stemming from  robustness.
Robustness enables a biological system to maintain function despite variations to some aspect of phenotypic form.
(Without robustness, overall function of an organism would be more likely to be negatively impacted by variation in a particular aspect of phenotypic form).

Although distinct from intraindividual degeneracy, intraindividual degeneracy might bolster interindividual degeneracy by providing bountiful opportunities for neutral variation to manifest.

\subsection{Relation to Evolvability}
Similar to how intraindividual degeneracy promotes individual evolvability, population degeneracy promotes population evolvability. Succinctly put, by providing a wider array of jumping-off points, the presence of structurally divergent individuals in a population increases the variety of phenotypic forms that can be realized in the offspring of that population.


\subsection{Example}
Earlobe attachment in \textit{Homo sapiens sapiens} provides an illustrative example of interindividual degeneracy. This phenotypic trait is distributed over a spectrum between attachment and detachment. The extremes of this spectrum are illustrated in Figure \inputandref{earlobe}. This trait is widely understood to be genetically determined \cite{Dutta1979EarlobeAssam}. This element of phenotypic diversity, which occurs in both forms within populations around the world, seems to be functionally neutral. That is, the trait does not affect how an individual interacts with his or her environment to determine his or her fitness. Hence, diversity in earlobe attachment represents interindividual degeneracy.


\section{Regularity} \label{sec:regularity}
\subsection{Definition}
Informally, regularity can be used to describe repetition phenotypic form. Repetitive form might manifest as symmetry and/or recurring  modular substructures. Formally, regularity refers to how much information is required to describe a structure \cite{Clune2011OnRegularity}.

\subsection{Relation to Evolvability}
A bias towards regularity in phenotypic form tends to represent a bias towards viable variation. That is, in many situations, regular phenotypes tend to outperform highly irregular phenotypes. This conclusion that regular phenotypes tend to be more viable, of course, depends on the demands of the environment that the phenotype inhabits. Phenotypic regularity tends to be useful in more regular environments (that is, environments that exhibit regular characteristics) \cite{Clune2011OnRegularity}. Many problem domains of interest to EANN researchers are highly regular \cite{Clune2011OnRegularity}. The natural world, the realm of flesh-and-blood creatures, also exhibits significant regularity \cite[pg 161]{Downing2015IntelligenceSystems}. Encodings biased towards regularity can also be thought of as promoting structures that were not directly selected for during evolution. These incidental phenotypic structures are called ``spandrels,'' referencing an architectural phenomenon that arises in a similarly unplanned for way: the awkward space between a curved archway set in a rectangular wall. These regular structures might be more likely to perform well in fitness cases that were not tested. It has been found that a tendency for regularity in evolving artificial neural networks produces networks that are capable of more generalized learning. That is, EANN that are more likely to successfully perform tasks beyond those they were explicitly tested on during evolution \cite{Tonelli2013OnNetworks}. Viewing learning as post-developmental (plastic) irregular refinement of regular neural structures created by development, a connection between regularity, irregular refinement (or plastic complexification), and plasticity is apparent in promoting phenotypic fitness.  In these ways, bias towards regularity increases evolvability because a higher proportion of individuals are viable.

\subsection{Example}
\textit{Aloe polyphylla}, which is known for the striking spiral arrangement of its leaves (Figure \inputandref{regularity_example}), exemplifies regularity in nature \cite{RoyalHorticulturalSocietyAloePolyphylla}. The phenotypic regularity exhibited by \textit{Aloe polyphylla} is an important adaptation. Phyllotaxis, the regular arrangement of leaves in plants, by minimizing the conflict between leaves for light, promotes efficient photosynthesis \cite{Kappraff2004GrowthNumber}.
