\chapter{Biological Perspective on Evolvability} \label{sec:biological_perspective_on_evolvability}

Evolutionary algorithms are ultimately inspired by biology; researchers in EANN would very much like to design systems with evolvability rivaling those of natural systems. The evolvability of biological systems is evidenced by the spectacular diversity of life, as well as the ability of populations to adapt to changing environmental conditions. Many of these ideas ultimately support ``facilitated variation'' \cite[pg. 40]{Downing2015IntelligenceSystems}, biasing novel variation towards potentially useful ends. However, a few instead are more selection-based, supporting the generation of variation in the first place. Let's take a tour of some of the mechanisms through which nature is thought to achieve evolvability (Chapter \ref{sec:definition}) and the traits that support evolvability (Section \ref{sec:concepts_related_to_evolvability}).


% \begin{itemize}
%     \item ``the organization of genotypic and phenotypic features into reusable components'' \cite{Reisinger2005TowardsEvolvability}
%     \item ``modular structures are those in which intracomponent acticity is more extensive than its intercomponent counterpart.'' \cite[p 207]{Downing2015IntelligenceSystems}
%     \item levels of modularity: ``a modular gene is one whose component base pairs are co-located [on the chromosome] and thus difficult to separate by the perturbing effect of crossover'', ``A genotype-phenotype mapping can exhibit varying degrees of modularity, inversely proportional to the pleiotropic interactions among genes'', ``modular phenotypic structures... can typically resist a perturbation's extent to the module itself'' \cite[p 208]{Downing2015IntelligenceSystems}
%     \item ``Once structures or mechanisms are consolidated and isolated (to some degree) from external influences, their probability of disruption declines and their potential for self-modification without global repercussions increases: they enhance both robustness and adaptability'' \cite[p 208]{Downing2015IntelligenceSystems}
%    \end{itemize}
        
\section{Duplication and Divergence} \label{sec:duplication_and_divergence}
Duplication and divergence provides a route to gradual complexification \cite[p 202]{Downing2015IntelligenceSystems}. This is a two step process. In the first step, the genetic information coding for a phenotypic feature is duplicated; this duplication of information is basically phenotypically neutral. In the second step, one of the copies of genetic information coding for a phenotypic feature is modified. This allows for re-use of functionality already developed by one system, making the introduction of sophisticated systems more tractable; evolution has already worked out most of the kinks. Further, this paradigm does not disrupt the functionality originally encoded in the genome; that information --- and, therefore, the process it describes --- is unchanged.

Homeobox genes provide a great example of duplicaiton and differentiation in action. These genes, which lay out the body plan of a developing embryo, have been duplicated and differentiated many times, at each step adding novel body plan features\cite[p 203]{Downing2015IntelligenceSystems}. These genes have seen re-use in determining brain segmentation. (It is interesting to note how the topology of the brain has many similarities to an animal body plan \cite[p 201]{Downing2015IntelligenceSystems}).
%   \begin{itemize}
%     \item ``structure A is duplicated yielding structure B, then B is free to change function without resulting in a loss of A'' \cite{Reisinger2005TowardsEvolvability}
%     \item idea: supports complexification process 
%     \item ``a copied gene (or gene group) gives evolution the flexibility to experiment with new functionalities (by mutating the copy) while retaining the original funcitonality. This is a low-risk route to complexification that begins with the production of an evolutionarily neutral copy that can eventually morph into a selectively advantageous trait.'' \cite[p 202]{Downing2015IntelligenceSystems}
%     \item example: homeobox genes, ``has duplicated and differentiated several times during evolution... each new Hox derivative provided by evolution yields new and varied segments along the phenotype'' 
%     \item fun example: the brain looks like a little squid! 
%   \end{itemize}
  
  
\section{Developmental Constraint} \label{sec:developmental_constraint}
Developmental constraint refers to the influence of the development process on the distribution of phenotypic forms that can be generated in the offspring of an individual \cite{Smith1985DevelopmentalBiology}. Developmental constraint might serve an adaptive purpose, biasing evolutionary search towards viable phenotypes \cite[pg 40]{Downing2015IntelligenceSystems} or might not serve an obvious adaptive purpose, such as an apparent constraint mandating that centipedes exhibit an odd number of body segments \cite{Arthur1999TheEvolution}. The concept of developmental constraint rests upon the idea that certain phenotypic forms are more likely to arise because they are more readily generated by physicochemical processes that underpin embryological development \cite{Smith1985DevelopmentalBiology}. Because the genotype is interpreted into phenotypic form in large part through the developmental system, the phenotypic outcomes of mutation are closely linked to the configuration of the developmental system. Developmental constraint is closely associated with the concept of canalization (Section \ref{sec:canalization}). Example \ref{sec:fly_symmetry}, which describes how developmental process enforces left-right symmetry in \textit{Drosophila melanogaster}, illustrates developmental constraint well.
 
        
\section{Hidden Genetic Variation} \label{sec:hidden_genetic_variation}
Hidden genetic variation, also known as cryptic variation or neutral variation, is a form of genotypic degeneracy. (Degeneracy is presented in Section \ref{sec:degeneracy}). In biological systems, many genotypes exist that maps to an identical or nearly identical phenotype. This may be due to the presence of variation non-coding DNA, variation in genes that are continually suppressed by regulatory mechanisms, degeneracy in the genetic code (i.e. several codons encoding the same amino acid residue), or simply encodes phenotypic traits that are nearly indistinguishable from a fitness point of view. It is thought that the homeostatic mechanisms that promote robustness facilitate hidden genetic variation by counteracting the phenotypic changes that might be induced by some forms of genetic variation \cite{Moczek2011TheInnovation}. Further, phenotypic plasticity (Section \ref{sec:plasticity}) might contribute to an intermediate form of cryptic variation where a trait is expressed in only a subset of the population due to environmental cues and is therefore somewhat hidden from selection \cite{Moczek2011TheInnovation}.
 
Because this cryptic variation is not selected upon by evolution, it can accumulate in a population. This accumulated heritable variation promotes evolutionary innovation \cite{Wilder2015ReconcilingEvolvability}. By allowing for a broader stable distribution of a population through a genotype space, cryptic variation increases the phenotypic diversity that can be realized in offspring from the population because individuals in the population can ``access radically different phenotypes in their immediate mutational neighborhood'' \cite{Wilder2015ReconcilingEvolvability}. Cryptic variation is also thought to allow for larger steps to be taken in the mutational search space during evolutionary search. Significant accumulated cryptic variation can rapidly switch to being expressed through a sensitizing mutation or environmental change \cite{Moczek2011TheInnovation}.
        
\section{Exploratory Growth}
Exploratory growth refers to the incorporation of search into the developmental process. Instead of having developmental components grow to hard-coded proportions and in hard-coded locations, developmental processes incorporate information about the the current state of the organism into the developmental trajectories of system components. Because other systems in a developing life form can change to adapt to changes in one system, exploratory growth reduces the probability of mutations leading to catastrophic fitness decline or outright mortality. Thus, exploratory growth promotes both canalization and robustness: translation of genetic changes into phenotypic effects are more likely to be resisted and, if they do occur, they are more likely to be viable because the development of the organism adapts to compensate for those changes. As Downing argues, due to exploratory growth ``the production of novel phenotypes does not require concerted change to many parts of the genome, a very low-probability combination of events, but rather a single change to a factor affecting an early phase of development. The rest just grow to fit the altered context...'' \cite[p 214]{Downing2015IntelligenceSystems}.

The structural components of the mammalian body develop in a ``grow to fit'' pattern. The development of bones and muscles is defined by a process in which they seek out attachment sites defined by other components of the developing system. These components ``grow to fit'' in the sense that they grow (or shrink) to fit an altered developmental context. The final form of these structures is determined through exploration of the developmental environment \cite[pg 214]{Downing2015IntelligenceSystems}. Mesychme cells, a cell type observed in animal embryos that is a precursor to several tissue types, provide another striking example of exploratory growth. During embryological cellular migration, mesenchyme cells extend filopodia to explore their environment; several filopodia attach to different sites on the blastocoel wall, and the final position of the cell is determined from a tug-of-war between the filopodia --- the mesenchyme cell will choose the site where it found most stable attachment. In this way, exploration performed by mesychyme cells contributes to the development process \cite[pg 214]{Downing2015IntelligenceSystems}.

The development of the nervous system is heavily marked by exploratory growth. In the brain, an excess of candidate neural networks are created during the development process. Fledgling networks with insufficient connectivity to the rest of the network are then culled; essentially, the final components of the brain are derived of a much larger number of competing trial subnetworks, many of which are failures \cite[p 214]{Downing2015IntelligenceSystems}. Ennervation of the body during development proceeds in a similar trial-and-error fashion. Excess neurons migrate and extend processes to seek out and compete for targets; many are unsuccessful and, ultimately, die \cite{Edelman2001DegeneracySystems}. 

\section{Weak Linkage}
Weak linkage is closely related to the concept of modularity, but a bit more specific. It describes the way that different systems interact. A distinction is drawn between instructive and enabling signals \cite[p 210]{Downing2015IntelligenceSystems}. Instructive signals contain significant amounts of information about the process to be performed, not just the information signaling that a process should be performed. Thus, instructive signals are contrary to modularity. In contrast, enabling signals are concise; the only information they contain is that the process should be performed. In a biological context, calcium signaling for muscle contraction would be considered an enabling signal. In an economic analogy, an instructive signal might be like a bakery receiving instructions from a central planning authority about the recipes, ingredient sources, quantities of goods to produce, and schedule for production that it should use; an enabling signal would be like a baker seeing the price of wheat flour increase and, in response, switching to using rice flour. Weak linkage supports evolvability because, with signaling simple, the chance of one system discovering how to signal another is higher so novel regulatory relations are more accessible to being stumbled upon. Of particular interest is the role of weak linkage in allowing an externally triggered signal to become innate; if a system is responding to simple signals, even if they are originally environmentally triggered, the simplicity of the signals makes them more accessible to being discovered by random variation and therefore more accessible to becoming innately triggered\cite[p 210]{Downing2015IntelligenceSystems}.
% \begin{itemize}
%   	\item instructive versus enabling signals; modular systems only require low-complexity external interactions in order to function properly; easy to go from externally triggered to innate via mutation 
%   \end{itemize}



\section{Baldwin Effect}
Learning, a type of phenotypic plasticity that provides another route to irregular refinement \cite{Clune2011OnRegularity}, is thought to bias evolutionary search towards discovering useful adaptation through a mechanism known as the Baldwin effect.\cite{Downing2010TheNetworks} By allowing a candidate solution to assume proximal phenotypic forms, learning enables evolutionary selection to act on information about a candidate solution's local phenotypic neighborhood. Through selective pressure for phenotypes proximal to high-fitness phenotypic forms, local phenotypic search ``buys evolutionary time'' until heritable scaffolding arises to support phenotypic adaptation originally attained via plasticity.\cite{Downing2010TheNetworks} The Baldwin effect is illustrated in Figure \inputandref{baldwin_effect}.