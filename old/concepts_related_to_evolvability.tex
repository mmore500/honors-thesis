\chapter{Concepts Related to Evolvability} \label{sec:concepts_related_to_evolvability}
This section introduces several traits associated with evolvability. Discussion of specific biological and evolutionary mechanisms that exhibit or induce these traits is provided in the following section, Section \ref{sec:biological_perspective_on_evolvability}. An overview of how these concepts interact is provided in 
\begin{samepage}
Figure \inputandref{mindmap}. Table \ref{table:intext} shows where these relationships occur in the text.
%\begingroup
\printindex[mindmap]{}
%\endgroup
\end{samepage}

\section{Modularity} \label{sec:modularity}
\subsection{Definition}
Modularity refers to the organization of a system into distinct systems where interaction between components of the system outstrips interactions between those components and outside elements \cite[p 207]{Downing2015IntelligenceSystems}. Modularity can also be viewed as, by distilling the system into compact units of independent functionality, allowing reuse of that functionality in different parts of the system through duplication \cite{Reisinger2005TowardsEvolvability}.

\subsection{Relation to Evolvability}
Downing argues convincingly for the utility of modularity,
\begin{displayquote}
Once structures or mechanisms are consolidated and isolated (to some degree) from external influences, their probability of disruption declines and their potential for self-modification without global repercussions increases: they enhance both robustness and adaptability \cite[p 208]{Downing2015IntelligenceSystems}
\end{displayquote}
The compartmentalization of functionality makes independent changes to one aspect of functionality possible while other aspects of functionality are preserved. By increasing the robustness of other aspects of functionality in an organism to changes in one aspect of functionality, modularity ultimately biases evolutionary search space towards viable offspring. \mindmapmark{\robustnessmodularity} Further, compartmentalization of functionality allows for re-use of that functionality in another context in the system. Because complete units of phenotypic functionality are more likely to be useful in new contexts than randomly generated structures, modularity allows for the generation of divergent phenotypes (i.e. diverse offspring) that are likely to be viable. In this way, modularity promotes complexification.\mindmapmark{\complexificationmodularity}

\subsection{Example}
The segregated cell roles in multicellular organisms provides an excellent example of modularity. Consider, in particular, simple squamous epithellium cells. These cells are found in areas requiring rapid diffusion or filtration [cite]. They appear in many, very different, anatomical contexts including the air sacs of lungs, the walls of capillaries, and the kidneys. Because these cells express a very specific set of genes, innovations can be specifically targeted to simple squamous epithellium cells by altering the regulatory balance of the gene network that these cells operate under. Although simple squamous epithellium cells share a large number of functional subunits --- proteins --- with other cell types, duplication and divergence (Section \ref{sec:duplication_and_divergence}) allows for the functional subunits of simple squamous epithellium cells to be modified individually. In fact, the commonality of proteins between cell types is another, yet deeper, form of modularity. The building blocks of cells --- such as the cellular machinery related to DNA replication, RNA transcription, and translation --- are discrete units of functionally preserved across all cell types.

\section{Robustness} \label{sec:robustness}
\subsection{Definition}
Robustness generally refers to the ability of a system's function to persist under perturbation. Biological systems exhibit robustness on three levels, as identified by Richter et al. \cite{Richter2015EvolvabilitySurvey}:
\begin{enumerate}
    \item system functionality is not degraded by stochastic fluctuations in the system, \label{item:robustness_stochastic}
  	\item phenotypic traits are generally not degraded by genetic variation, and \label{item:robustness_genetic}
    \item the phenotype is able to function if environment changes. \label{item:robustness_environment}
\end{enumerate}
  
\subsection{Relation to Evolvability}
By reducing outright mortality and other phenotypic variations with catastrophic fitness implications, robustness biases evolutionary search towards useful variation.\mindmapmark{\canalizationrobustness}

\subsection{Example}
The heart provides a prime example of level \ref{item:robustness_stochastic} robustness. The waves of electric potential that govern the function of the heart are generally robust to perturbation; the heart can usually recover from momentary disruptions to these waves and resume normal function after a short time. However, the heart is not a totally robust organ: it can be susceptible to ventricular fibrillation, where normal heart function is essentially halted by the emergence of a spiral wave pattern of electrical activation [cite]. In many cases, the consequences of the heart not being able to recover from the perturbation that induced the spiral wave pattern of activation are fatal.

Duplicate genes exemplify robustness at level \ref{item:robustness_genetic} \cite{Gu2003EvolutionMutations}. It has been shown that deletion duplicate genes have a significantly lower proportion of fatal outcomes and a significantly higher proportion of weak or no effect outcomes in \textit{Saccharomyces cerevisiae} (yeast) \cite{Gu2003EvolutionMutations}. Robustness is also provided by alternate metabolic pathways and regulatory networks that can compensate for the absence of a gene \cite{Gu2003EvolutionMutations}.

Finally, robustness at level \ref{item:robustness_environment} is exhibited by the brown rat (\textit{Rattus norvegicus}). This creature has been wildly successful in a wide range of environments --- today, its range spans nearly the entire globe --- that differ vastly from the environment where it originated \cite{Wikipediacontributors2016BrownRat}. In fact, \textit{Rattus norvegicus} is particularly notorious for thriving in urban areas, a completely novel environment that only came into existence in their current form over the last millennium or so. The brown rat displays impressive robustness to environmental variation.

\section{Canalization} \label{sec:canalization}
\subsection{Definition}
Canalization is closely related to the concept of robustness, but focuses less on only introducing bias towards preserving function and more about the bias in variation that can be actualized, the kind of variation that is ``allowed.'' Reisinger et al. succinctly describe the phenomenon of canalization as ``transforming random mutations into structured phenotypic variation'' \cite{Reisinger2007AcquiringRepresentations}. Canalization can be thought of as an evolutionary bias towards exploring viable regions of the search space, a bias that emerges from the evolutionary process itself.

\subsection{Relation to Evolvability}
By biasing the evolutionary search towards viable variation, canalization increases the quality of the population that evolutionary selection works on \cite[p 40]{Downing2015IntelligenceSystems}.\mindmapmark{\usefulvariationcanalization} Thus, canalization makes the evolutionary search process more efficient, more likely to discover useful variation that, ultimately, leads to the production of higher fitness solutions.

\subsection{Example} \label{sec:fly_symmetry}
Experiments performed by Tuinstra et al. illustrate the effects of canalization in relation to symmetry in \textit{Drosophilia melanogaster} \cite{Tuinstra1990LackDevelopment}. Unlike experiments employing artificial selection for other symmetrically distributed traits (i.e. overall eye size, overall bristle count, etc.), selection for left-right asymmetric thorasic bristle number was to no avail; although phenotypic variation in left-right thorasic bristle number existed in the population, that variation was not heritable. Tuinstra et al. hypothesize that this canalization results from the developmental process, ``as no evidence is available for an independent left-right gradient in the embryo, quantitative traits can only be expressed variably along an existing gradient of positional information or a morphagen`` \cite{Tuinstra1990LackDevelopment}. Similar results have been reported in relation to artificial selection for asymmetric distribution of eye size \cite{Coyne1987LackMelanogaster}, eye facet number \cite{ManyardSmith1960ThePattern}, and wing-folding behavior \cite{Purnell1973SelectionMelanogaster} in \textit{Drosophilia melanogaster}. These experiments are illustrated in Figure \inputandref{canalization_example}.

%   \begin{itemize}
%     \item ``organizing the effects of mutations such that some features become more resistant to change'' \cite{Reisinger2005TowardsEvolvability}
%     \item ``transforming random mutations into structured phenotypic variation'' \cite{Reisinger2007AcquiringRepresentations}
%     \item ``Representational canalization, i.e. the degree to which genotypic changes affect the phenotype, can be measured by averaging phenotypic differences after a fixed number of mutations.'' \cite{Reisinger2007AcquiringRepresentations}
%     \item ``Another indicator of a representation's canalization is how much fitness changes in response to mutation'' \cite{Reisinger2007AcquiringRepresentations}
%    \end{itemize}
   
\section{Complexification} \label{sec:complexification}
\subsection{Definition}
Complexification refers to the development of sophisticated phenotypic functionality through the incremental refinement and elaboration over evolutionary time \cite[pg 202]{Downing2015IntelligenceSystems}. In this process, a phenotypic structure is results not from a single mutation event but instead from a sequential series of modifications, each building upon the last; many intermediate forms of the phenotypic feature, with a trend towards increasing sophistication, are observed. In respect to evolving artificial neural networks, complexification might refer to the network starting out with very few nodes and having nodes added --- thus, developing a more complex phenotype ---  over the course of evolution \cite{Clune2011OnRegularity}.

The stepping stones of complexification, intermediate phenotypic forms, need not share the same functionality. It is entirely possible for an existing structure to be appropriated for a new purpose, a phenomenon referred to as exaptation \cite{Gould1982Exaptation-aForm} in the case where the original phenotypic trait is replaced and referred to as neofunctionalization in the case where the original phenotypic trait is preserved \cite{Escriva2006NeofunctionalizationReceptors}.  

\subsection{Relation to Evolvability}
Complexification biases evolutionary search towards useful variation. In short, variants of already-existing phenotypic structures are more likely to be useful than randomly-generated phenotypic structures.\mindmapmark{\usefulvariationcomplexification} If existing phenotypic structures could not readily be elaborated upon, the chance of evolution happening upon useful variation would be vanishingly small.

\subsection{Example}
The evolution of the vertebrate eye epitomizes complexification. It is thought that this structure evolved through a series of intermediates, beginning with a simple region of enervated photosensitive cells. A folded-in,  photosensitive pouch-like structure, which provided directional sensitivity, is thought to have arisen next. Pinhole and lens structures, which provide greater visual acuity, are thought to have descended from the pouch structures \cite{Gregory2008TheOrgans}. These intermediate phenotypic structures, each elaborating on an existing phenotypic form, can be observed in extant organisms as illustrated in Figure \inputandref{eye_evolution}.

The evolution of flight provides another characteristic example of complexification. Unlike the eye, where intermediate structures all performed the same task (that is, sensing electromagnetic waves from the surrounding environment), the evolution of flight is marked by profound repurposing of feathers. It is thought that feathers originally evolved as thermal insulation and were only later repurposed for flight \cite{Gould1982Exaptation-aForm}. Although serving an unrelated purpose, insulatory feathers provided a jumping off point for flight feathers.

\section{Plasticity} \label{sec:plasticity}
\subsection{Definition}
Plasticity refers to environmental influence on the phenotype. In biology, environmental and genetic influences, together, shape the phenotype. Environmental influences may alter the trajectory of the developmental process or may otherwise induce phenotype changes in response to environmental stimulus \cite{Fusco2010PhenotypicConcepts}. Further distinctions between can be drawn between direct and indirect plasticity. In the first, environmental influence is exerted directly on developmental or physiological processes. In the second, environmental signals prompt responses that are mediated by physiological or developmental systems; that is, cues from the environment are processed more like informational signals than coercive physical influence \cite{Fusco2010PhenotypicConcepts}.

\subsection{Relation to Evolvability}

The exact role of phenotypic plasticity in evolution is an issue of active debate in the evolutionary biology community \cite{Pigliucci2008IsEvolvable}. However, several hypotheses describing how phenotypic plasticity might relate to evolution and evolvability have been put forward. Phenotypic plasticity might serve as a kind of local exploration of the phenotypic search space, allowing for the immediate expression of a phenotype with increased fitness and for biasing the evolutionary search towards high-fitness regions of the search space \cite{Downing2012HeterochronousBaldwinism}. It is also thought that the homeostatic mechanisms that mediate an organism's interactions with its environment might promote robustness \cite{Moczek2011TheInnovation}.\mindmapmark{\robustnessplasticity} Researchers have suggested that phenotypic modularity might promote plasticity, especially in plants \cite{Schlichting1986ThePlants, DeKroon2005APlants}. Thus, selection for plasticity might promote modularity.\mindmapmark{\modularityplasticity} In these ways, plasticity might promote useful variability.

Conditional expression of phenotypic traits through plasticity allows for relaxed selection on the genotypic locus determining those traits. Thus, significant genetic variation can accumulate at that locus in a population.\mindmapmark{\interindividualdegeneracyrobustness} In a process known as genetic accommodation, the environmental influence on when rarely-expressed phenotypic traits are expressed can be diminished or erased through sensitizing mutation; what once was induced via environmental signal can become constitutive. Such processes have been observed experimentally via artificial selection \cite{Moczek2011TheInnovation}.

It is known that plasticity plays an important role in adaptation to unpredictable or variable environment \cite{Fusco2010PhenotypicConcepts}. Finally, the role plastic processes such as learning might play in concert with the regularity --- i.e. providing a mechanism of irregular refinement of highly regular phenotypic structures generated via development --- is an open question  \cite{Clune2011OnRegularity}.\mindmapmark{\usefulvariationregularitycomplexificationplasticity}\mindmapmark{\usefulvariationregularitycomplexification}

\subsection{Example}
 
Tadpoles of \textit{Spea multiplicata}, which is also known as the Mexican spadefoot toad, develop along two alternate trajectories in response to environmental signals. The two tadpole morphs of \textit{Spea multiplicata}, depicted in Figure \inputandref{spea_multiplicata} exhibit different jaw and digestive tract configurations and prefer different diets \cite{Fusco2010PhenotypicConcepts}. One morph is specially suited to an carnivorous diet while the other is suited to an omnivorous one\cite{Pfennig1992PolyphenismStrategy}.

While \textit{Spea multiplicata} illustrate phenotypic plasticity in development, phenotypic plasticity --- although often more subtle --- also manifests on an ongoing basis throughout the lifespan of most creatures. \textit{E. coli} provide a textbook example of phenotypic alteration in response to environmental stimulus. These bacteria selectively express an enzyme used in the digestion of lactose, $\beta$-galactosidase in the presence of lactose when alternate food sources, such as glucose, are unavailable. This phenotypic change is mutable; in the absence of lactose or the presence of glucose production of $\beta$-galactosidase halts \cite[page XXX]{Griffiths2015IntroductionAnalysis}.

Plasticity, or course, is one of the most fascinating traits of biological brains. Although the layout of the brain is largely shaped by the genetic influence, its function is heavily marked by environmental. Operant conditioning is a classic example of neural adaptation in response to environmental stimulus. In these experiments, laboratory animals --- typically rats --- are rewarded for performing a certain action in response to a stimulus  \cite{Skinner1991TheAnalysis}. For example, a rat might be conditioned to pull a lever in response to flashing light in order to earn a pellet reward. Thus, the brain of the brain of the rat, which controls its actions, has been altered to promote the appropriate response to the stimulus. Neuroplasticity is also observed at a larger scale. Brain regions in humans are known to exhibit hyper or hypotrophy in response to extended, extreme use or disuse. For example, blindness is thought to induce hypotrophy of the occipital regions (visual processing elements that are less heavily used). Compensatory hypertrophy is observed in brain regions that, due to blindness, are more heavily relied upon or are relied upon in new ways \cite{Lepore2010BrainSubjects}. 

\section{Degeneracy} \label{sec:degeneracy}

Degeneracy refers to the exhibition of similar functionality by structurally different systems  \cite{Edelman2001DegeneracySystems}. For example, one might refer to push doors, rotating doors, and sliding doors as degenerate; although they all accomplish similar tasks, they are structurally dissimilar. Although degeneracy is usually treated on the level of an individual organism, I expand the concept discuss degeneracy among biological populations and between populations.

\subsection{Intraindividual Degeneracy}

\paragraph{Definition}
The concept of intraindividual degeneracy is closely related to complexification and plasticity (Sections \ref{sec:complexification} and \ref{sec:plasticity}). In a degenerate system, distinct components with different structures perform the same function; these components are redundant. However, under different circumstances --- such as unusual environmental stress --- the functionality of these components might diverge. \cite{Richter2015EvolvabilitySurvey}

\paragraph{Relation to Evolvability}
Intraindividual degeneracy is thought to be key to the evolvability observed in biology. Intraindividual degeneracy provides a solution to the seemingly paradoxical relationship between robustness (Section \ref{sec:robustness}) and evolutionary innovation, the ability to generate heritable novel variation \cite{Whitacre2010Degeneracy:Evolvability}.\mindmapmark{\individualevolvabilityrobustness}\mindmapmark{\individualevolvabilityrobustnessintraindividualdegeneracy} How can a system have both robustness, which constrains change to the system, while also innovating? Intraindividual degeneracy seems to be the answer. By having multiple independent components that perform the same task, the system is secured against the failure of one of the components and, therefore, more likely to persist under extreme environmental conditions due to the diverse response of the degenerate components to those conditions \cite{Whitacre2010Degeneracy:Evolvability}.\mindmapmark{\robustnessintraindividualdegeneracy} However, intraindividual degeneracy also provides a diverse set of phenotypic structures that can be repurposed or elaborated upon. Essentially, intraindividual degeneracy protects against loss of function mutations, which are often deleterious, while providing ample jumping off points to generate novel phenotypic function. In this way, intraindividual degeneracy promotes individual evolvability. \mindmapmark{\individualevolvabilityintraindividualdegeneracy}

\paragraph{Example}
A diverse set of mechanisms effect synaptic efficacy in mammalian brains. In a presynaptic neurons, synaptic efficacy is known to be increased via increases in the number of synaptic vesicles, the concentration of neurotransmitters in those vesicles, and an increase in the sensitivity of mechanisms triggering those vesicles to release. In the postsynaptic neuron, several different kinds of neurotransmitter receptors --- which are further diversified by differing phosphorylation states --- affect postsynaptic response. The response can be modulated by alterations to the quantity, types, and phosphorylation of these receptors. Postsynaptic response is further modulated by changes the quantity, identity, and distribution of ion channels and pumps in the cellular membrane \cite{Edelman2001DegeneracySystems}. 

\subsection{Interindividual Degeneracy}

\paragraph{Definition}

I define interindividual degeneracy as the presence of genetically and/or phenotypically dissimilar individuals in a population that, despite those dissimilarities, interact with their environment in identical or near-identical ways (i.e. are functionally . Interindividual degeneracy might also be cast as diversity, although the term population degeneracy focuses more directly upon the fact that, although structurally divergent, individuals in the population are functionally similar. Interindividual degeneracy might manifest as hidden genetic variation (Section \ref{sec:hidden_genetic_variation}) or neutral variation. Although distinct from intraindividual degeneracy, intraindividual degeneracy might bolster interindividual degeneracy by providing bountiful opportunities for hidden or neutral variation to manifest.  \mindmapmark{\interindividualdegeneracyintraindividualdegeneracy}


\paragraph{Relation to Evolvability}
Similar to how intraindividual degeneracy promotes individual evolvability, population degeneracy promotes population evolvability. Succinctly put, by providing a wider array of jumping-off points, the presence of structurally divergent individuals in a population increases the variety of phenotypic forms that can be realized in the offspring of that population.
\mindmapmark{\populationevolvabilityinterindividualdegeneracy}

\paragraph{Example}
Earlobe attachment in \textit{Homo sapiens sapiens} provides an illustrative example of interindividual degeneracy. This phenotypic trait is distributed over a spectrum between attachment and detachment. The extremes of this spectrum are illustrated in Figure \inputandref{earlobe}. This trait is widely understood to be genetically determined \cite{Dutta1979EarlobeAssam}. This element of phenotypic diversity, which occurs in both forms within populations around the world, seems to be functionally neutral. That is, the trait does not affect how an individual interacts with his or her environment to determine his or her fitness. Hence, diversity in earlobe attachment represents interindividual degeneracy.

\subsection{Fitness Degeneracy}

\paragraph{Definition}
I define fitness degeneracy as the ability of divergent strategies to yield fitness in a particular environment. This might be recast as the the ability of individuals with functionally different phenotypes to both attain high fitness. In the simplest terms, fitness degeneracy refers to the idea that there might be different ways to succeed in a particular environment. This closely corresponds to the biological concept of niches, although it considers the opportunity an environment might provide for divergent phenotypic functionality to succeed inside a species as well as between them.

\paragraph{Relation to Evolvability}
By allowing for functionally divergent phenotypes to succeed, fitness degeneracy promotes the accumulation of diversity within a population. This diversity broadens the range of phenotypic forms that are accessible to the offspring of a population, thus promoting population evolvability \mindmapmark{\populationevolvabilityfitnessdegeneracy}. The coexistence of speciated populations within the same ecosystem that exploit radically different strategies to earn their metabolic keep  
can also be seen as fitness degeneracy. This degeneracy, the ability of radically different fitness strategies to coexist within the biosphere, has vastly expanded the breadth of phenotypic form explored by evolution. 
\begin{displayquote}
 A related approach is to set a vast number of separate, diverse evolutionary objectives, which approximates the infinite number of niches created by the ever-changing natural world [27].  Such optimization leads to frequent goal-switching, as lineages fit on one objective invade other objectives, which rewards lineages that produce behaviorally diverse offspring and increases evolvability
\cite{Mengistu2016EvolvabilityIt}
\end{displayquote}

\paragraph{Example}
The male mating patterns of \textit{Uta stansburiana}, a lizard found in the coast range of California, illustrates fitness degeneracy --- how different strategies stemming from phenotypic diversity can be successful in terms of fitness. Male \textit{Uta stansburiana} are found in three morphs: orange, blue, and yellow \cite{Sinervo1996TheStrategies}. The orange morph is characterized as ``hyper-masculine.'' It staunchly defends its territory and will battle male trespassers. Lizards of the yellow morph are ``sneakers.'' They pass themselves off as females to get past other males. The blue morph, lying somewhere in between the other two morphs, is described as the ``mate-guarding'' morph. Blue morph lizards are more circumspect to combat and not as easily deceived by interloping yellow morphs. Figure \inputandref{lizards} provides a visual overview of how these morphs interact. Much like the game of rock, paper, scissors, each morph outcompetes a morph and is itself outcompeted by another morph. No single phenotypic strategy is favored by the environment; although divergent, each lizard morph exhibits fitness.

\section{Temporally Varying Goals} \label{sec:tvg}
\subsection{Definition}
Most biological organisms do not evolve in a static environment. Instead, their environment changes over evolutionary time. These changes might be due to geophysical factors, such as changes in the climate or the geological composition of an area. These changes might also be due to biological factors, through evolutionary changes in the organisms with which they cooperate, compete, or otherwise interact. Thus the fitness-related goals of these organisms --- that is, the specifications that they must meet in order to succeed --- are temporally varying.

\subsection{Relation to Evolvability}

Evolutionary simulations have revealed that gradual changes to the environment  --- that is, a temporally varying fitness function --- promote desirable characteristics related to evolvability including robustness, modularity, and individual evolvability \cite{Kashtan2007VaryingEvolution, Wilder2015ReconcilingEvolvability}. It is thought that a gradually shifting fitness function might induce evolutionary pressure for these traits, essentially providing a means of selecting for them. Robustness would be useful in a temporally varying goals scheme, allowing to persist under environmental conditions different from those of their ancestors.\mindmapmark{\robustnesstemporallyvaryinggoals} Similarly, modularity, which allows for the repurposing of evolved units of functionality and reduces the impact of changes in one aspect of functionality on other aspects of functionality, would also prove useful in responding to evolutionary pressure for constant, gradual adaptation to changing environmental conditions.\mindmapmark{\modularitytemporallyvaryinggoals} Kashtan et al. point out that modularity is a particularly important trait because, throughout changes in environmental conditions, the necessity for many aspects of functionality is preserved.

\begin{displayquote}
On the level of the organism, for example, the same subgoals, such as feeding, mating, and moving, must be fulfilled in each new environment but with different nuances and combinations. On the level of cells, the same subgoals such as adhesion and signaling must be fulfilled in each tissue type but with different input and output signals. On the level of proteins, the same subgoals, such as enzymatic activity, binding to other proteins, regulatory input domains, etc., are shared by many proteins but with different combinations in each case \cite{Kashtan2007VaryingEvolution}.
\end{displayquote}

In fact, phenotypic modularity has been observed to spontaneous in artificial evolution experiments performed with a temporally varying fitness function \cite{Kashtan2007VaryingEvolution}. Finally, under a temporally varying goals regimen, individuals that are predisposed to yielding variable offspring are advantaged over individuals that do not.\mindmapmark{\individualevolvabilitytemporallyvaryinggoals} Although much of that variation is likely not to be useful, some of it is likely to be and will allow to exert dominance over individuals without fresh adaptation to the changing environmental conditions. As Wilder, et al. put it, ``if selection sets a moving target, individuals will be more likely to introduce variation in their offspring to adapt to an uncertain future; mutations to the genotype will be more likely to result in phenotypic change'' \cite{Wilder2015ReconcilingEvolvability}. By inducing a selective pressure for individuals capable of generating relatively swift adaptive change to track changing environmental conditions, temporally varying goals promote a number of essential traits related to evolvability.

\subsection{Example}

Let us discuss the example illustrated in Figure   \inputandref{hummingbird_selection_pressure}. This example examines a hypothetical population  of hummingbirds, which feed on purple flowers. In order to successfully feed, hummingbird beak lengths must match the length of their food source --- beaks can be neither too short nor too long or the hummingbird will be unable to feed effectively. Suppose that the lengths of the purple flowers that the hummingbirds depend were to be manipulated over evolutionary time, proceeding through cycles of gradual increase and decrease. One might expect to see, compared to a population of hummingbirds that exist in a static environment, greater variability in beak length between individuals in the same generation (i.e. individual evolvabiltiy), increased segregation of developmental processes that determine beak length from other developmental processes (i.e. modularity), and/or greater ability of the hummingbird to persist with limited nutritional resources (i.e. robustness). (Experimental results evidencing the effects of temporally varying goals in artificial evolution are described in Section \ref{sec:mvff}).


\section{Regularity} \label{sec:regularity}
\subsection{Definition}
Informally, regularity can be used to describe repetition phenotypic form. Repetitive form might manifest as symmetry and/or recurring  modular substructures. Formally, regularity refers to how much information is required to describe a structure \cite{Clune2011OnRegularity}.

\subsection{Relation to Evolvability}
A bias towards regularity in phenotypic form tends to represent a bias towards viable variation.\mindmapmark{\usefulvariationregularity} That is, in many situations, regular phenotypes tend to outperform highly irregular phenotypes. This conclusion that regular phenotypes tend to be more viable, of course, depends on the demands of the environment that the phenotype inhabits. Phenotypic regularity tends to be useful in more regular environments (that is, environments that exhibit regular characteristics) \cite{Clune2011OnRegularity}. Many problem domains of interest to EANN researchers are highly regular \cite{Clune2011OnRegularity}. The natural world, the realm of flesh-and-blood creatures, also exhibits significant regularity \cite[pg 161]{Downing2015IntelligenceSystems}. Encodings biased towards regularity can also be thought of as promoting structures that were not directly selected for during evolution. These incidental phenotypic structures are called ``spandrels,'' referencing an architectural phenomenon that arises in a similarly unplanned for way: the awkward space between a curved archway set in a rectangular wall. These regular structures might be more likely to perform well in fitness cases that were not tested. It has been found that a tendency for regularity in evolving artificial neural networks produces networks that are capable of more generalized learning. That is, EANN that are more likely to successfully perform tasks beyond those they were explicitly tested on during evolution \cite{Tonelli2013OnNetworks}. Viewing learning as post-developmental (plastic) irregular refinement of regular neural structures created by development, a connection between regularity, irregular refinement (or plastic complexification), and plasticity is apparent in promoting phenotypic fitness.\mindmapmark{\usefulvariationregularitycomplexificationplasticity}  In these ways, bias towards regularity increases evolvability because a higher proportion of individuals are viable.

\subsection{Example}
\textit{Aloe polyphylla}, which is known for the striking spiral arrangement of its leaves (Figure \inputandref{regularity_example}), exemplifies regularity in nature \cite{RoyalHorticulturalSocietyAloePolyphylla}. The phenotypic regularity exhibited by \textit{Aloe polyphylla} is an important adaptation. Phyllotaxis, the regular arrangement of leaves in plants, by minimizing the conflict between leaves for light, promotes efficient photosynthesis \cite{Kappraff2004GrowthNumber}.
