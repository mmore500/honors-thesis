\paragraph{Irregularity}
  \begin{itemize}
    \item adaptation potential: essentially, butting up against structural restrictions of genotype mapping (i.e. regularity)
    \item ``During the development of a complex system some parts may already be (close-to) optimal in the current state, while other parts have to adapt further to specific demands or changing conditions... A representation that can only change both targets simultaneously is counterproductive.'' \cite{Richter2015EvolvabilitySurvey}
    \item ``HybrID that allows the HyperNEAT indirect encoding to produce regular patterns in concert with a direct encoding that can modify these patterns to produce irregularities. HybrID matches HyperNEAT’s performance on regular problems, but outperforms HyperNEAT on problems with irregularity, which demonstrates that HyperNEAT struggles to generate certain kinds of irregularity on its own. The success of HybrID raises the interesting question of whether indirect encodings may truly excel not as stand-alone algorithms, but in combination with a further process that refines their regular patterns'' \cite{Clune2011OnRegularity}
  \end{itemize}
  
  
% as the representation's ability to ``learn'' the fitness bias of an environment and , the ability to ``manipulate its own bias over the course of evolution'' \cite{Reisinger2007AcquiringRepresentations}. When viewing evolvability in this manner, an important distinction can be drawn between latent evolvability and acquired evolvability. According to Reisinger et al., latent evolvability describes ``the representation’s underlying capacity for becoming evolvable'' while acquired evolvability describes ``evolvability learned in response to a particular fitness function'' \cite{Reisinger2005TowardsEvolvability}.

% \begin{itemize}
%   \item amount of information able to acquire (different speeds of varying fitness function)\cite{Reisinger2005TowardsEvolvability}
%   \item RMS distance in behavioral characterization space (population diversity), ``We approximate an individual’s capacity to generate
% future phenotypic variation by measuring phenotypic
% variability among a sample of the individual’s simulated off-
% spring (which are discarded). Such variability is quantified as
% the number of unique behaviors; in particular, each offspring
% is considered sequentially and added to a list of unique behaviors
% only if its behavior is significantly different from the
% behaviors of organisms already in the list. Two behaviors are
% considered different if the distance between them according
% to a domain-specific behavioral distance metric is above a
% pre-specified threshold'' \cite{Mengistu2016EvolvabilityIt}
%   \item ``a measurement of evolvability should characterize the
% amount of variability that can be accessed in an individual or population's genetic neighborhood; number of distinct phenotypes in a genetic neighborhood around individual; amounts to Monte Carlo sampling of the phenotypic space surrounding an individual \cite{Wilder2015ReconcilingEvolvability} change
% \end{itemize}

% \cite{Reisinger2005TowardsEvolvability} \cite{Mengistu2016EvolvabilityIt} \cite{Wilder2015ReconcilingEvolvability} \cite{Tarapore2015EvolvabilityBenchmarks}

% \begin{itemize}
%   \item idea: the value of a solution isn't just its ability to generate good fitness scores, but the ability to innovate and adapt in evolutionary time $\rightarrow$ question: how could natural selection ``favor properties that may prove useful to a given lineage in the future, but have no present adaptive function''?, ``teleological fallacy'', multiple levels of selection (i.e. species selection in addition to individual selection?) \cite{Pigliucci2008IsEvolvable}

%   \item individual evolvability: ability to generate a diverse set of offspring from an individual ``behavioral diversity of its immediate  offspring,  and  select  organisms  with  increased offspring variation.'' \cite{Mengistu2016EvolvabilityIt}

%   \item population evolvability
%   \begin{itemize}
%     \item ``It is important to note that population-level evolvability is not equal to the sum over individual evolvability because the novel phenotypes contributed by different individuals may be redundant'' \cite{Wilder2015ReconcilingEvolvability}
%     \item ``On the one hand, evolvable individuals are more likely than others to introduce phenotypic variation in their offspring. On the other hand, in evolvable populations a greater amount of phenotypic variation is accessible to the population as a whole, regardless of how evolvable any individual may be in isolation'' \cite{Wilder2015ReconcilingEvolvability}
%   \end{itemize}

%   \item ability to ``learn'' bias of environment and to canalize offspring \cite{Reisinger2005TowardsEvolvability} (to ``manipulate own bias over the course of evolution'' \cite{Reisinger2006SelectingRepresentations})
%   \begin{itemize}
%     \item  ``latent evolvability will be used to describe the representation’s underlying capacity for becoming evolvable'' \cite{Reisinger2005TowardsEvolvability}
%     \item ``acquired evolvability will be used to refer to its evolvability learned in response to a particular fitness function'' \cite{Reisinger2005TowardsEvolvability}
%     \item ``the evolvability of a genome can be approximated with the fitness of the local mutation landscape around that genome'' \cite{Reisinger2007AcquiringRepresentations}

%   \end{itemize}
% \end{itemize}
